\documentclass{article}
\usepackage[utf8]{inputenc}
\usepackage{enumitem}

\title{The Upsilon Pi Epsilon – Marquette University Constitution}
\date{2009}

\begin{document}

\maketitle

\section*{Preamble}
Upsilon Pi Epsilon – Marquette University seeks to recognize and reward academic excellence in
the computing and information disciplines. In order to uphold this goal we declare this to be the
Upsilon Pi Epsilon – Marquette University Constitution.

\section{Name}
The name of this organization shall be Upsilon Pi Epsilon – Marquette University, hereafter
referred to as the Organization

\section{Purpose}
The purpose of this Organization shall be to honor high scholarship in the computing and information disciplines and support continued research in those fields.

\section{Membership}
\subsection{Regular Membership}
Regular membership shall be open to any full time Marquette University
undergraduate student with a minimum 3.0 GPA and majoring in a computer
related field and having received nominations from their faculty and/or peers.
\subsection{Associate Membership}
Associate membership shall be open to any part time student, graduate student,
professional student, faculty member, staff member, or administrator at
Marquette University working or studying in a computer related field.
\subsection{Non-discrimination Clause}
Consistent with all applicable federal and state laws and University policies, this Organization and its subordinate bodies and
officers shall not discriminate on the basis of race, color, age, sexual orientation, religion, Veteran’s status, sex, national origin, or disability in its selection of members, educational programs, or activities.
\subsection{Dues}
Dues include a one-time \$80 payment.

\section{Officers}
\subsection{Positions}
Officers of the Organization shall be as follows: President, Vice President, and Secretary/Treasurer.
\subsection{Elections}
Election of officers occurs during the early spring. Elections will be held by all active members. Terms are twelve months in duration.
\subsection{Terms}
Officers shall take office in the fall and shall serve for a period of twelve months.
\subsection{Officer Eligibility}
Officers shall not be on academic or university probation at the time of their elections and throughout their terms of office.
\subsection{Duties of Officers}
\begin{enumerate}[label=(\Alph*)]
\item The President is responsible for conducting organization business.
\item Checks must be co-signed by the Treasurer and by either the President or
Vice-President.
\item The officer responsible for conducting meetings is the President.
\end{enumerate}

\section{Removal of Officers}
\subsection{Removal}
Officers failing to fulfill the given responsibilities and duties may be removed by the regular members of the Organization.
\subsection{Procedures for Removal}
The removal of an officer requires a 2/3 vote of a quorum following the
notification of the officer in question. Such notification shall be provided in
writing no less than seven working days prior to the vote.

\section{Replacement of Officers}
\subsection{President Vacancy}
In the case where the Presidential Office is vacant, the Vice-President will immediately fill the position.
\subsection{General Vacancy}
All other executive board positions found to be vacant shall be filled by election immediately.

\section{Meetings}
\subsection{Frequency of Meetings}
A regularly scheduled general meeting shall be held at least once a semester. The officers may call additional meetings when the need arises.
\subsection{Quorum}
A quorum shall consist of 50\% of the regular members.
\subsection{Requirement of Quorum}
A quorum shall be present in order for any official business to be conducted.
Official business shall include elections of officers, setting of dues and any other major decisions affecting the Organization.
\subsection{Parliamentary Rules}
Parliamentary Authority used is \underline{\textit{Robert’s Rules of Order, Newly Revised}}.

\section{Committees}
\subsection{Committee Creation}
The officers of the Organization shall have the authority to create any
committees, standing or special, that will further the purpose of the
organization.
\subsection{Nominating Committee}
The Nominating Committee, run by the President, shall be in charge of reviewing candidates for future membership.

\section{Affiliation}
This Organization shall be affiliated with Upsilon Pi Epsilon and shall abide by its constitution and by-laws in all cases where there is not conflict between their constitution and by-laws and this constitution and/or the rules, regulations, or policies of Marquette University. In instances of conflict, this constitution and/or rules, regulations or policies of Marquette University shall take precedence over the constitution or by-laws of Upsilon Pi Epsilon.

\section{Amendments}
\subsection{Notice for Amendments}
All amendments to this constitution require notice of one week prior to being discussed and voted upon.
\subsection{Quorum for Amendments}
All amendments require a 2/3 vote of a quorum for adoption.
\subsection{Amendment Approval}
Amendments become effective only after approval by both the Office of Student
Development and the Student Senate of the Marquette University Student
Government (MUSG).

\end{document}
